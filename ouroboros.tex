\documentclass[11pt]{article}

\usepackage{amsmath, amssymb, amsthm}
\usepackage{geometry}
\usepackage{hyperref}
\usepackage{bm}
\usepackage{parskip}

\geometry{margin=1.2in}

\title{\textbf{Ouroboros}\\[0.4em] \large Neural Cryptanalysis of Linear Feedback Shift Registers}
\author{Stochastic Batman}
\date{February 18, 2026}

\begin{document}
	\maketitle
	
	\section{Linear Feedback Shift Registers}
	
	\subsection*{Definition}
	
	A \textbf{Linear Feedback Shift Register} (LFSR) of degree $n$ is a finite-state machine whose state at time $t$ is a vector of bits $$ \mathbf{s}^{(t)} = \bigl(s_1^{(t)},\, s_2^{(t)},\, \ldots,\, s_n^{(t)}\bigr) \in \mathbb{F}_2^n $$ where $\mathbb{F}_2 = \{0, 1\}$ is the field of integers modulo 2, with addition defined as XOR ($\oplus$) and multiplication as AND ($\land$).
	
	The state evolves by a linear recurrence over $\mathbb{F}_2$: $$ s_n^{(t+1)} = c_1 s_1^{(t)} \oplus c_2 s_2^{(t)} \oplus \cdots
	\oplus c_n s_n^{(t)}, \qquad c_i \in \mathbb{F}_2 $$
	and then the register shifts: $s_i^{(t+1)} = s_{i+1}^{(t)}$ for $i < n$. The output (keystream) at each step is the bit that falls off the end: $$ b^{(t)} = s_1^{(t)}. $$
	
	\subsection*{Characteristic Polynomial and Taps}
	
	The recurrence is encoded by the \emph{characteristic polynomial} $$ p(x) = x^n + c_{n-1}x^{n-1} + \cdots + c_1 x + 1 \in \mathbb{F}_2[x] $$
	The non-zero coefficients indicate which bit positions feed back into the computation; these positions are called \textbf{taps}.
	
	\href{https://github.com/Stochastic-Batman/Ouroboros.git}{\textbf{Ouroboros}} uses a degree-32 LFSR with the maximal-length polynomial $$p(x) = x^{32} + x^{22} + x^2 + x + 1, $$ giving taps at positions $\{32,\,22,\,2,\,1\}$.
	
	\subsection*{Maximal Length and Periodicity}
	
	If $p(x)$ is \emph{primitive}(can produce every single non-zero element in that field through repeated multiplication) over $\mathbb{F}_2$, the LFSR cycles through every non-zero state exactly once before repeating. The period is then $$ T = 2^n - 1 $$
	For $n = 32$ this gives $T = 4,294,967,295 \approx 4.3 \times 10^9$ bits - astronomically long, yet the entire sequence is determined by the $n$-bit seed and the tap set.
	
	\subsection*{State Transition as a Matrix}
	
	Over $\mathbb{F}_2$ the one-step transition is a linear map. Writing the state as a column vector, the update rule is $$ \mathbf{s}^{(t+1)} = A\,\mathbf{s}^{(t)}, \quad A \in \mathbb{F}_2^{n\times n}, $$ where $A$ is the \emph{companion matrix} of $p(x)$:
	$$ A =
	\begin{pmatrix}
		0 & 1 & 0 & \cdots & 0 \\
		0 & 0 & 1 & \cdots & 0 \\
		\vdots & & & \ddots & \vdots \\
		0 & 0 & 0 & \cdots & 1 \\
		c_1 & c_2 & c_3 & \cdots & c_n
	\end{pmatrix}. $$
	After $k$ steps: $\mathbf{s}^{(t+k)} = A^k\,\mathbf{s}^{(t)}$, all arithmetic mod 2. This is why the sequence is completely predictable given the seed and tap set: the RNN's job is to \emph{infer} this linear structure from the raw bitstream alone.
	
	\subsection*{The Berlekamp-Massey Theorem}
	
	A classical result states that $2n$ consecutive output bits are sufficient to reconstruct both the tap polynomial and the internal state of any degree-$n$ LFSR exactly. For our 32-bit register, 64 bits of observed output are \emph{provably enough} to break it analytically. The neural approach instead attempts to learn this structure implicitly via gradient descent.
	
	\section{Recurrent Neural Networks: A Recap}
	
	\subsection*{The Elman RNN}
	
	Given a sequence of inputs $x_1, x_2, \ldots, x_T$ (scalars here, since each input is one bit), an \textbf{Elman RNN} maintains a hidden state $\mathbf{h}_t \in \mathbb{R}^H$ and produces an output $y_t \in \mathbb{R}$ at each step according to:
	$$ \mathbf{h}_t = \tanh (W_{xh} x_t + W_{hh} \mathbf{h}_{t-1} + \mathbf{b}_h) $$
	$$ y_t = \sigma (W_{hy} \mathbf{h}_t + b_y) $$
	where the learnable parameters are:
	\begin{align*}
		W_{xh} &\in \mathbb{R}^{H\times 1},\quad
		W_{hh} \in \mathbb{R}^{H\times H},\quad
		\mathbf{b}_h \in \mathbb{R}^H, \\
		W_{hy} &\in \mathbb{R}^{1\times H},\quad
		b_y \in \mathbb{R}.
	\end{align*}
	The initial state is set to $\mathbf{h}_0 = \mathbf{0}$.
	
	\subsection*{Activation Functions}
	
	\textbf{Hyperbolic tangent} is used for the hidden layer because it is zero-centered, has gradient in $(-1,1)$, and gives the network the ability to represent both excitation and inhibition: $$ \tanh(z) = \frac{e^z - e^{-z}}{e^z + e^{-z}}. $$
	
	\textbf{Sigmoid} is used at the output to produce a valid probability: $$ \sigma(z) = \frac{1}{1 + e^{-z}} \in (0,1). $$
	
	\subsection*{Loss Function}
	
	Since the network predicts the probability that the next bit is 1, we use \textbf{Binary Cross-Entropy} (BCE):
	$$ \mathcal{L}(y_t, \hat{y}_t) = -\Bigl[y_t \log \hat{y}_t + (1-y_t)\log(1-\hat{y}_t)\Bigr], $$ where $y_t \in \{0,1\}$ is the true next bit and $\hat{y}_t$ is the predicted probability. The total loss over a sequence of length $T$ is $$ \mathcal{L}_{\text{total}} = \frac{1}{T}\sum_{t=1}^{T} \mathcal{L}(y_t, \hat{y}_t). $$
	
	\section{Backpropagation Through Time (BPTT)}
	
	\subsection*{Overview}
	
	BPTT unrolls the RNN across $T$ time steps and treats the result as a deep feedforward network, then applies the chain rule. Gradients flow backwards from the loss at each step all the way to the parameters.
	
	\subsection*{Output Layer Gradient}
	
	The fused BCE + sigmoid gradient has a particularly clean form. Let
	$z_t = W_{hy}\mathbf{h}_t + b_y$ so that $\hat{y}_t = \sigma(z_t)$. Then: $$ \frac{\partial \mathcal{L}}{\partial z_t} = \hat{y}_t - y_t. $$
	
	This is the gradient used directly in the code as \texttt{dy}.
	
	From this, the parameter gradients for the output layer are:
	\begin{align*}
		\frac{\partial \mathcal{L}}{\partial W_{hy}} &= (\hat{y}_t - y_t)\,\mathbf{h}_t^\top, \\
		\frac{\partial \mathcal{L}}{\partial b_y}    &= \hat{y}_t - y_t.
	\end{align*}
	
	\subsection*{Hidden Layer Gradient}
	
	The gradient of the loss w.r.t.\ $\mathbf{h}_t$ receives two contributions: one from the output at step $t$, and one propagated from step $t+1$: $$\frac{\partial \mathcal{L}}{\partial \mathbf{h}_t}
	= W_{hy}^\top\,(\hat{y}_t - y_t)
	+ W_{hh}^\top\,\boldsymbol{\delta}_{t+1}, $$
	where $\boldsymbol{\delta}_t$ is the \emph{delta} flowing into the hidden pre-activation.
	
	Since $\mathbf{h}_t = \tanh(\mathbf{a}_t)$ with $\mathbf{a}_t$ the pre-activation, the chain rule through tanh gives: $$ \boldsymbol{\delta}_t = \frac{\partial \mathcal{L}}{\partial \mathbf{h}_t} \odot \bigl(1 - \mathbf{h}_t^2\bigr), $$ where $\odot$ denotes element-wise multiplication and $1 - \mathbf{h}_t^2$ is the element-wise derivative of $\tanh$.
	
	The remaining parameter gradients follow:
	\begin{align*}
		\frac{\partial \mathcal{L}}{\partial W_{xh}} &= \boldsymbol{\delta}_t\, x_t^\top, \\
		\frac{\partial \mathcal{L}}{\partial W_{hh}} &= \boldsymbol{\delta}_t\, \mathbf{h}_{t-1}^\top, \\
		\frac{\partial \mathcal{L}}{\partial \mathbf{b}_h} &= \boldsymbol{\delta}_t, \\
		\frac{\partial \mathcal{L}}{\partial \mathbf{h}_{t-1}} &= W_{hh}^\top\,\boldsymbol{\delta}_t
		\quad\leftarrow\text{propagated to previous step}.
	\end{align*}
	
	Gradients accumulate by summing across all $T$ steps before updating weights.
	
	\subsection*{Gradient Clipping}
	
	Repeated matrix products $W_{hh}^T$ in deep unrollings cause gradients to grow exponentially with $T$ (\emph{exploding gradients}). The code applies element-wise clipping: $$g \leftarrow \operatorname{clip}(g,\,-c,\,c)
	= \max \bigl(-c,\,\min(c,\,g)\bigr),\quad c = 5.0. $$
	
	\subsection*{Weight Initialisation}
	
	Weights are initialised with \textbf{Xavier (Glorot) uniform} initialisation: $$W_{ij} \sim \mathcal{U} \left(-\frac{1}{\sqrt{n_\text{in}}}, \frac{1}{\sqrt{n_\text{in}}}\right), $$ where $n_\text{in}$ is the fan-in of the layer. This keeps the variance of activations roughly constant across layers at initialisation, avoiding saturation of tanh from the first forward pass.
	
	\subsection*{Parameter Update (SGD)}
	
	After accumulating and clipping gradients, a vanilla SGD step is applied: $$ \theta \leftarrow \theta - \eta\,\nabla_\theta\mathcal{L},
	\quad \eta = 0.005. $$
	
	\section{The Cryptanalysis Objective}
	
	Let $\mathcal{S} = (b^{(0)}, b^{(1)}, b^{(2)}, \ldots)$ be the LFSR keystream. The network is trained on the supervised task $$ \hat{y}_t \approx \mathbb{P} \left( b^{(t+1)} = 1 \mid b^{(0)}, \ldots, b^{(t)} \right). $$ Since the LFSR is deterministic, this probability is degenerate: it is either 0 or 1. Perfect prediction corresponds to the network having implicitly learned the characteristic polynomial $p(x)$ and the tap set $\{32, 22, 2, 1\}$ from raw observations alone.
	
	The hidden state $\mathbf{h}_t \in \mathbb{R}^H$ can be interpreted as the network's learned \emph{proxy} for the LFSR's internal state $\mathbf{s}^{(t)} \in \mathbb{F}_2^{32}$. If $H \geq 32$, there is sufficient capacity to represent the full register, and we would expect accuracy to approach 100\% given enough training.
\end{document}